\nonstopmode
\documentclass[10pt, a4paper]{article}
\parindent=20pt
\parskip=8pt
\usepackage[spanish]{babel}
\usepackage[utf8]{inputenc}	
\usepackage{framed, color}
\definecolor{shadecolor}{gray}{0.95}
\usepackage{amsmath,amsfonts,amsthm}				
\usepackage[pdftex]{graphicx}								% Enable pdflatex
\usepackage[svgnames]{xcolor}						
\usepackage{geometry}
	\textheight=700px								
\usepackage{url}									
\usepackage{wrapfig}								
\usepackage{bbding}	
\textheight = 650pt
\footskip = 30pt

\author{Seguridad Android.}
\date{}								% Symbols
\begin{document}

\thispagestyle{empty}

\begin{center}

\vspace{2cm}

Departamento de Computación,\\
$22^{\circ}$ Escuela de Verano de Ciencias Informáticas,\\
Universidad Nacional de Río Cuarto

\vspace{4cm}

\begin{Huge}
Seguridad y Protección de datos en aplicaciones Android\end{Huge}

\vspace{0.5cm}

\begin{huge}
Trabajo Práctico Final
\end{huge}

\vspace{1cm}

Febrero de 2015

\vspace{8cm}


\vspace{1cm}

\begin{tabular}{|c|c|c|}
\hline
Apellido y Nombre & Universidad & E-mail\\
\hline
Izcovich, Sabrina     & FCEyN - UBA & sizcovich@gmail.com\\
Vita, Sebastián       & FCEyN - UBA & sebastian\_vita@yahoo.com.ar\\
\hline
\end{tabular}

\end{center}

\newpage
\tableofcontents
\newpage

\section{Introducción}
En el siguiente trabajo práctico, debimos analizar la aplicación Android $BadBrowser$ con el fin de hallar sus vulnerabilidades. La misma se trata de un navegador que permite sincronizar el historial de un usuario registrado con un servidor externo. Dicha aplicación contiene vulnerabilidades del tipo de las vistas en el curso, que pueden ser:
\begin{itemize}
\item \textbf{LOCALES}
\begin{itemize}
\item \textbf{DATOS MAL PROTEGIDOS}
\item \textbf{COMPONENTES MAL EXPORTADOS}
\end{itemize}
\item \textbf{DE RED}
\begin{itemize}
\item \textbf{ACTIVAS}
\item \textbf{PASIVAS}
\end{itemize}
\item \textbf{REMOTAS}
\end{itemize}

Con el fin de hallarlas, debimos realizar un análisis dinámico y otro estático de la aplicación. Para esto último, fue necesario descargar la última actualización de $android-apktool$\footnote{https://code.google.com/p/android-apktool/} (v2.0.0RC4) dentro de la máquina virtual provista por la cátedra como también decompilar la aplicación para acceder a su código fuente.
Para una correcta experimentación, se nos proveyó de los fuentes del servidor externo para que pudiera ser corrido utilizando python.

\newpage
\section{Análisis}

\subsection{Manifest}

En primer lugar, decidimos analizar el $Manifest$, archivo en el que se enumeran las actividades, los servicios, los receptores, los proveedores de contenidos y los permisos requeridos por la aplicación. 

Las configuraciones encontradas fueron las siguientes:
\begin{itemize}
\item \begin{verbatim}<permission android:name="ar.sadosky.badbrowser.permission.SyncBrowser"/> \end{verbatim} 
\item \begin{verbatim}<permission
android:name="ar.sadosky.badbrowser.permission.WRITE"
android:protectionLevel="signature"/> \end{verbatim}

Estos dos permisos son exportados por la aplicación. El primero permite sincronizar los datos del browser utilizando la cuenta creada. El segundo indica que el nivel de protección es de tipo ``signature'', lo que significa que es sólo accesible a aplicaciones firmadas con el mismo certificado que la que registra el permiso.

\item \begin{verbatim} application android:allowBackup="true" \end{verbatim}

Esta configuración presenta una vulnerabilidad de tipo \textbf{Almacenamiento inseguro de datos privados} pues se almacena la información sin ningún tipo de encriptación. Esto se considera un problema de seguridad pues una persona podría realizar un backup de la aplicación por medio de $adb$ y obtener datos privados de la misma en su computadora.


\item \begin{verbatim}
<provider android:authorities="ar.sadosky.browser.history_provider"
android:exported="true"
android:name="ar.sadosky.badbrowser.HistoryProvider"
android:writePermission="ar.sadosky.badbrowser.permission.WRITE"/>
\end{verbatim}

La configuración anterior le da al provider el permiso de ser exportado públicamente (a través de $exported=``true''$).
Dado que no se exhibe ninguna configuración sobre los permisos de tipo lectura, no se impide que las aplicaciones externas puedan leer el historial de los usuarios, presentando una vulnerabilidad de tipo \textbf{Componentes mal exportados}. Para impedir este tipo de vulnerabilidades, es necesario declarar explícitamente la actividad como privada con $android:exported=``false''$.

\item \begin{verbatim}
<intent-filter>
<data android:scheme="javascript"/>
\end{verbatim}
La configuración anterior permite inyectar Javascript en un contexto determinado. Dicha configuración le da la posibilidad a cualquier aplicación maliciosa de correr JS para robar cookies. Por ejemplo, un posible ataque consiste en enviar un Intent para cargar la página víctima (como www.google.com) y, luego, enviar otro Intent utilizando el esquema javascript para correr JS en el contexto de esa página.
\end{itemize}

Consideramos que el resto de las configuraciones están bien adaptadas a los fines de la aplicación y no presentan relevancia, por lo que decidimos no mencionarlas.

\newpage
\subsection{Browser Server}
Luego, procedimos analizando el $browser\_server.py$ con el fin de hallar las posibles fallas presentes en el mismo:

En primer lugar, cada $login$ que realiza la aplicación, genera y guarda en un conjunto un string de identificación de sesión ($session\_key$) que se envía codificado al usuario utilizando AES como algoritmo de encriptación. Dado que no hay ningún tipo de mecanismo en el código para caducar o borrar las sesiones activas, un atacante podría generar múltiples logins de forma masiva sin que la aplicación rechace los pedidos, pudiendo realizar un DOS hasta llenar la memoria RAM del servidor.

Por otra parte, al registrarse, el servidor guarda el $uuid$ y la $password$ del usuario sin ninguna medida de seguridad o verificación de operaciones de registros masivos por parte de los usuarios. Del mismo modo, tampoco se controla que se realicen distintos registros utilizando el mismo $uuid$.

A continuación, se plantean los siguientes ataques:
\begin{shaded}
\begin{itemize}
\item Al no verificar ni el largo de los campos ingresado ni la cantidad de registros que se realiza, un atacante podría realizar un DOS llenando la memoria del servidor, encargado de guardar los registros.
\item Un atacante podría sobrescribir el registro de otro usuario utilizando su mismo $uuid$ y con la posibilidad de cambiar su contraseña.
\end{itemize}
\end{shaded}

Otra vulnerabilidad ocurre al realizar peticiones a $/sync$ POST y GET. Tanto para el $/sync$ POST como para el GET, sólo se verifica que la $session\_key$ exista en el conjunto de keys activas almacenadas en el servidor. Este error da la posibilidad a un atacante de subir (utilizando POST) o leer (utilizando GET) el contenido de cualquier otro usuario por medio de una $session\_key$ utilizada anteriormente que, como se mencionó antes, nunca caducan.

Se pueden plantear los siguentes ataques:
\begin{shaded}
\begin{itemize}
\item Un atacante podría utilizar su $session\_key$ o capturar un mensaje a $/sync$ y reutilizarlo cambiando el campo $email$ que identifica al dueño del contenido a leer o subir. De este modo, puede leer el contenido guardado por cualquier usuario o cambiar el campo referido al contenido de lo que será guardado y sobrescribir el contenido guardado por otro usuario.
\item Un atacante puede utilizar una $session\_key$ válida de las maneras nombradas anteriormente y realizar una petición GET a $/sync$ donde el campo mail sea el del usuario al que quiere atacar pudiendo obtener, así, el contenido de su historial sincronizado.
\item Al no tener ningún tipo de verificación al cargar el contenido, podría realizarse un DOS enviando mensajes POST de forma masiva a $/sync$ con alguna $session\_key$ válida, mails que no se encuentren registrados y contenido pesado para terminar con el agotamiento de la memoria del servidor.
\end{itemize}
\end{shaded}

\newpage


Luego del seguimiento realizado, nos focalizamos en la búsqueda de vulnerabilidades a través del estudio de los distintos componentes de la aplicación. Los riesgos detectados fueron los que siguen:

\begin{itemize}

\item \textbf{Mal uso de criptografía:} Al observar los \texbf{def encrypt(self,value)} y \texbf{def decrypt(self,value)}, notamos que la clave secreta del cipher simétrico se encuentra hardcodeada. Usualmente, las mismas se generan de forma $random$ a través de alguna función provista por el lenguaje. 
Por otro lado, dado que el cipher está en modo CBC, la reutilización de un IV puede facilitar información sobre el primer bloque de texto plano, el largo y cualquier prefijo compartido entre dos mensajes.
Si no se cambia el IV en cada sesión, los datos están mayormente sujetos a ataques capaces de revelar la clave. De este modo, hallamos que el diseño de proceso de encriptación es débil.

Este tipo de problemas podría generar el siguiente escenario:

Supongamos que somos víctimas de un ataque de tipo ``Man in the middle''. Dado que la $session\_key$ se encripta con el mismo par (IV, clave) y la información que se transfiere durante la conexión puede repetirse, alcanza con un análisis estadístico criptográfico para conocer la clave secreta de encriptación, para luego, desencriptar el mensaje del usuario.

\item \textbf{Incorrecta validación del servidor:} El servidor mantiene un bajo nivel de verificación para todos los servicios que ofrece. Como mencionamos anteriormente, el registro de un usuario puede sobrescribirse con sólo volver a realizar el proceso de registro con el mismo $uuid$, pudiendo usar la identidad de un usuario y cambiarle la contraseña. Dado que el servicio de sincronización no comprueba la identidad de los usuarios, se puede utilizar una $session_key$ válida para modificar y leer los historiales de los usuarios.

Por otro lado, la incapacidad del servidor por detectar intentos masivos de registros o de inicios de sesión puede provocar una serie de DOS descritos anteriormente.

\item \textbf{Vulnerabilidad de autenticación/validación de usuario:} Ejecutando la aplicación, encontramos que no hace falta conocer la clave de un usuario para loguearse sino que alcanza con poner un $uuid$ válido o una clave válida. Este error puede verse en la siguiente línea de código:
\begin{verbatim}
(email_key['uuid'] == _uuid or email_key['password'] == password )
\end{verbatim}
donde el ``or'' muestra que alcanza con cualquiera de esos dos campos para ingresar a la aplicación.
Por otro lado, el $uuid$ 0000000000000000 es siempre válido.

\item \textbf{Autenticación/autorización pobre:} Dado que la cuenta no se bloquea tras X intentos de sesión fallidos, se podrían probar distintas claves indefinidamente, dando la posibilidad a un atacante a intentar con contraseñas hasta hallar la correcta. 

\item \textbf{Transmisión insegura de datos:} No se utiliza SSL ni TLS como protocolos criptográficos para proporcionar comunicaciones seguras por la red, generando una transmisión insegura de los datos. Esto se debe a que en vez de correr sobre HTTPS, el servidor corre sobre HTTP, dando la posibilidad de sniffear (\textbf{ataque pasivo}) y/o modificar (\textbf{ataque activo}) los datos.

\item \textbf{Ataque de red activo:} Debido a la falta de cifrado en el uso de peticiones POST, se puede interceptar y alterar el historial enviado en una petición POST a $/sync$ a través de un ataque Man in The Middle.

\item \textbf{Ataque de red pasivo:} Como se expuso anteriormente, dada la falta de cifrado en el uso del protocolo web, es posible sniffear y ver los datos/historial de una persona, ya sea con su $uuid$ como con su $password$. Para esto, basta con utilizar un analizador de red, como ser $Scapy$ o $Wirsehark$ y capturar toda la información que se intercambia con el servidor.

\item \textbf{Datos mal protegidos:} La información se envía encriptada con una clave que se encuentra hardcodeada en el código, luego se la encripta con base 64 por ser transportada por HTTP. Por lo tanto, todos los datos que se envían a través del browser pueden ser desencriptados fácilmente. 

\item \textbf{Datos mal protegidos:} Analizando el código, descubrimos que el historial de los usuarios se almacena en el mismo archivo consecutivamente, por lo que cualquier usuario podría leer el historial de los demás.


\item \textbf{Almacenamiento incorrecto de datos:} Cuando un usuario guarda un sitio al que ingresa con HTTPS en el historial, el mismo se guarda con HTTP entonces, cuando vuelve a ingresar a dicho sitio, entra con HTTP.

Por ejemplo, imaginemos que almacenamos en el historial el sitio https://www.facebook.com. El mismo se guarda de la forma http://www.facebook.com luego, la próxima vez que accedamos a dicho sitio, lo haremos de forma no segura.

\item \textbf{Vulnerabilidad de tipo Remota:} Analizando el archivo $BrowserWebViewClient$, encontramos que en la función $shouldOverrideUrlLoading()$ se utiliza el esquema ``file://'', luego, cualquier Javascript de archivos corriendo en el contexto de dicho esquema puede acceder al contenido de cualquier otro origen. Dicha característica puede permitir al atacante forzar el Webview para abrir un archivo malicioso como, por ejemplo, uno en la SD. De este modo, un atacante podría inyectar una cookie con Javascript y redirigir a badbrowser al archivo que almacena las cookies. Dicha vulnerabilidad es considerada de \textbf{Webview}.
\end{itemize}

\newpage
\section{Prueba de Concepto}

Una posible prueba de concepto para mostrar una vulnerabilidad de tipo local consiste en realizar un $adb$ $query$ con el fin de obtener el historial del usuario del teléfono:

Para ello, es menester correr la aplicación y, en una terminal, escribir el siguiente comando:

\begin{verbatim}
adb shell content query --uri "content://ar.sadosky.browser.history_provider/history"
\end{verbatim}

Otra forma de acceder al historial del usuario de forma dinámica es, desde la aplicación, escribir en el buscador del navegador:
\begin{verbatim}
file:///data/data/ar.sadosky.badbrowser/databases/history.db
\end{verbatim}

De este modo, se accede al archivo interno de almacenamiento del historial y el mismo es mostrado por pantalla.

Como se puede observar, es muy simple acceder al historial del usuario sin necesidad de ningún tipo de información, exponiendo la vulnerabilidad local de $BadBrowser$.


\newpage
\section{Conclusión}

A partir de un exhaustivo análisis dinámico y estático de la aplicación $BadBrowser$, encontramos distintos problemas de seguridad. Entre ellos, se pueden ver algunos de tipo transmisión insegura, almacenamiento inseguro de datos privados, autenticación/autorización pobres y mal uso de criptografía. 

Entre las vulnerabilidades descubiertas, notamos que la aplicación se comunica con servidores externos utilizando un protocolo inseguro y exporta servicios de manera incorrecta, pudiento los atacantes:
\begin{itemize}
\item Interferir y recolectar información privada por los protocolos sin encriptación.
\item Alterar y/o conseguir el historial privado de los usuarios de forma remota sin una dificultad aparente de forma remota.
\item Conseguir el historial de los usuarios de forma local utilizando ADB o alguna aplicación, mediante el content provider mal exportado.
\end{itemize}

Concluimos además, que el cifrado utilizado para la $session\_key$ puede ser vulnerado debido a que la utilización de un único IV puede permitir ataques estadísticos criptográficos que podrían resultar en la obtención de la clave secreta del cifrado.

\end{document}