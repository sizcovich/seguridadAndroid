
\documentclass[paper=a4,fontsize=11pt]{scrartcl}	 			% KOMA-article class
\usepackage[utf8]{inputenc}
\usepackage[spanish]{babel}			
%\usepackage[english]{babel}								% English language/hyphenation
%\usepackage[protrusion=true,expansion=true]{microtype}		% Better typography
\usepackage{amsmath,amsfonts,amsthm}					% Math packages
\usepackage[pdftex]{graphicx}								% Enable pdflatex
\usepackage[svgnames]{xcolor}							% Colors by their 'svgnames'
\usepackage{geometry}
	\textheight=700px									% Saving trees ;-) 
\usepackage{url}										% Clickable URL's
\usepackage{wrapfig}									% Wrap text along figures
\frenchspacing									% Better looking spacings after periods
\pagestyle{empty}								% No pagenumbers/headers/footers
\usepackage{bbding}									% Symbols
\begin{document}

En el siguiente trabajo práctico, debimos analizar la aplicación Android $BadBrowser$ con el fin de hallar sus vulnerabilidades . 
Como trabajo final para aprobar el curso se entregará a los alumnos una aplicación Android. Esta aplicación contiene vulnerabilidades del tipo de las vistas en el curso. Se trata de un navegador que permite sincronizar el historial de un usuario registrado con un servidor externo. Además de la aplicación Android, van a tener los fuentes del servidor externo así lo pueden correr en sus redes utilizando python.

\end{document}